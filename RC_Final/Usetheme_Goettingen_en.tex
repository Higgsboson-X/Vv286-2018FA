\documentclass[hyperref={pdfpagelabels=true}]{beamer}
%\usepackage{etex} % Math fonts misbehave
\usepackage{lmodern} 
%\usefonttheme{structurebold}

\usepackage{gauss}
\usepackage{amsmath}
\usepackage{amsfonts}
\usepackage{lipsum}
\usepackage{graphicx}
\usepackage{subfigure}
\usepackage{setspace}
\usepackage{amssymb}
\usepackage{multirow}
\usepackage{xcolor}
\usepackage{ragged2e}
\usepackage{mathrsfs}
\usepackage{bbold}
\usepackage[all]{xy}

\setbeamertemplate{footline}[frame number]
\setbeamercolor{normal text}{bg=white!10}
\setbeamertemplate{theorems}[numbered]

\newcommand{\C}{\mathbb{C}} \newcommand{\F}{\mathbb{F}} \newcommand{\R}{\mathbb{R}} \newcommand{\Q}{\mathbb{Q}}
\newcommand{\N}{\mathbb{N}}
\newcommand{\Z}{\mathbb{Z}}

\newcommand{\HRule}{\rule{\linewidth}{0.5mm}}
\newcommand{\U}{\mathrm}
\newcommand{\celsius}{\ensuremath{^{\circ}\mathrm{C}}}
\newcommand{\myseries}[2]{#1_1,\ldots,#1_{#2}}
\newcommand{\highlightr}[1]{\textcolor[rgb]{1,0.3,0.2}{\emph{\textbf{#1}}}}
\newcommand{\highlightg}[1]{\textcolor[rgb]{0.1,0.5,0.3}{\emph{\textbf{#1}}}}
\newcommand{\structb}[1]{\textcolor[rgb]{0.2,0.2,0.7}{#1}}
\newcommand{\bemph}[1]{\emph{\textbf{#1}}}
\newcommand{\tabincell}[2]{\begin{tabular}{@{}#1@{}}#2\end{tabular}}


\renewcommand\arraystretch{1.8}


\title{Honors Mathematics \uppercase\expandafter{\romannumeral4}\\Final Review}
\author{CHEN Xiwen} 
\date{\today} 
\institute[UM-JI]{UM-SJTU Joint Institute}


\AtBeginSubsection[]{
\begin{frame}
\tableofcontents[sectionstyle=show/shaded,subsectionstyle=show/shaded]
\end{frame}
}

\begin{document}



\begin{frame}
\titlepage
\end{frame} 


\begin{frame}
\frametitle{Table of contents}
\tableofcontents
\end{frame} 


\section{Separation of Variables for PDEs}


\subsection{Uniqueness of Solution}


\begin{frame}{Uniqueness of Solution}

\justifying
\structb{Example 1.} Prove the uniqueness of solution of the three dimensional wave on $\Omega\subset\R^3$
\begin{align*}
c^2u_{tt} = \Delta u
\end{align*}
which satisfy the boundary conditions
\begin{align*}
u(x, y, z, t) = F(x, y, z, t), \qquad (x, y, z)\in \partial \Omega
\end{align*}
and initial conditions
\begin{align*}
u(x, y, z, 0) = G(x, y, z), \qquad u_t(x, y, z, 0) = H(x, y, z).
\end{align*}
\highlightr{Hint.} Green's identity
\begin{align*}
\int_{\Omega}\langle \nabla u, \nabla v\rangle d\tau = -\int_{\Omega} u\cdot\Delta v d\tau + \int_{\partial \Omega^*} u\frac{\partial v}{\partial n}dA.
\end{align*}

\end{frame}


\subsection{Solving PDEs}

\begin{frame}{Separation of Variables for PDEs}

\begin{enumerate}
	\justifying
	\item Make ansatz
	$$u(x_1, \ldots, x_n) = u_1(x_1)\cdot u_2(x_2)\cdots u_n(x_n).$$
	\item Write out the boundary conditions with the ansatz. (Usually this will reduce to initial conditions for the ODE with only one variable.)
	\item Solve the ODEs one-by-one. (Begin with the one that has boundary conditions.)
	\begin{itemize}
		\justifying
		\item Obtain ODE in the form $Lu = \lambda u$ with boundary condition.
		\item Solve this ODE and obtain eigenvalues \& eigenfunctions.
		\item Plug in these eigenvalues to solve other equations.
	\end{itemize}
	\item Gather all the eigenfunctions for each variable to obtain the general solution of the PDE.
	\item Fit the general solution into the initial conditions.
	\begin{itemize}
		\justifying
		\item The eigenfunctions for one ODE normally form an orthonormal system. (Usually Fourier series or Bessel functions.)
		\item Expand the initial condition with this orthonormal system.
	\end{itemize}
\end{enumerate}

\end{frame}

\begin{frame}{Orthonormal Systems}

\justifying
\structb{Fitting initial conditions.} There are (most commonly) two series that we can expand the initial conditions to. On the interval $[0, L]$, we have
\begin{itemize}
\justifying
\item Fourier series (usually cosine or sine series).
\begin{align*}
\left\{\frac{1}{\sqrt{L}}, \sqrt{\frac{2}{L}}\cos\left(\frac{\pi nx}{L} \right) \right\}^{\infty}_{n=1}, \quad \left\{\sqrt{\frac{2}{L}}\sin\left(\frac{\pi nx}{L} \right) \right\}^{\infty}_{n=1}.
\end{align*}
\item Bessel functions. Expand the initial condition into
\begin{align*}
\left\{\frac{1}{\sqrt{L}|J'_n(\alpha_{n, m})|} J_n(\alpha_{n, m}\sqrt{x/L}) \right\}_{m=1}^{\infty}
\end{align*}
so that
$$f(x) = \sum_{m=1}^{\infty}\frac{1}{L\cdot J'_n(\alpha_{n, m})^2}\langle J_n(\alpha_{n, m}\sqrt{(\cdot)/L}), f\rangle J_n(\alpha_{n, m}\sqrt{x/L}).$$
\end{itemize}

\end{frame}


\subsection{Sturm-Liouville Problems Derived from PDE}


\begin{frame}{Inhomogeneous PDEs}

\justifying
\begin{block}{}
	\justifying
	\structb{Step 2.} \highlightg{Solve the ODEs one-by-one. (Begin with the one that has boundary conditions.)} 
\end{block}

\begin{block}{}
	\structb{Example 2.} Solve the inhomogeneous heat equation
	\begin{align*}
	u_{xx} - u_t = -2x, \qquad (x, t)\in (0, 1)\times \R_{+}
	\end{align*}
	with Dirichlet boundary conditions
	\begin{align*}
	u(0, t) = 0, \qquad u(1, t) = 0, \qquad t > 0
	\end{align*}
	and initial temperature distribution
	\begin{align*}
	u(x, 0) = x - x^2, \qquad x\in [0, 1].
	\end{align*}
	
\end{block}

\end{frame}




\subsection{Orthogonality of Eigenfunctions}



\begin{frame}{Orthogonality of Eigenfunctions}

\begin{block}{}
	\justifying
	\structb{Step 3.} \highlightg{Solve this ODE and obtain eigenvalues \& eigenfunctions.}
\end{block}
\begin{block}{}
	\justifying
	The choice of basis largely depends on the eigenfunctions obtained from the Sturm-Liouville problem.
\end{block}


\end{frame}




\begin{frame}{Mixed Boundary Conditions}

\justifying
\begin{block}{}
	\justifying
	\structb{Step 3.} \highlightg{Solve this ODE and obtain eigenvalues \& eigenfunctions.}
\end{block}
\begin{block}{}
	\structb{Example 3.} Show how a solution to the heat equation
	\begin{align*}
	u_{xx} - u_t = 0, \qquad (x, t)\in (0, 1)\times \R_{+}
	\end{align*}
	with mixed boundary conditions
	\begin{align*}
	u(0, t) = 0,\qquad u_x(1, t) = 0, \qquad t>0
	\end{align*}
	and initial temperature distribution
	\begin{align*}
	u(x, 0) = f(x), \qquad x\in [0, 1]
	\end{align*}
	can be obtained.
\end{block}


\end{frame}





\begin{frame}{Orthogonality of Eigenfunctions}

\begin{block}{}
	\justifying
	\structb{Step 3.} \highlightg{Solve this ODE and obtain eigenvalues \& eigenfunctions.}
\end{block}
\begin{block}{}
	\structb{Example 4 (RC 9).} Solve the wave equation problem
	\begin{align*}
	4u_{tt} & = u_{xx}, \\
	u_x(-\pi, t) & = u_x(\pi, t) = 0, \quad u(x, 0) = x^2, \quad u_t(x, 0) = 0.
	\end{align*}
	The basis is chosen as
	\begin{align*}
	\left\{\frac{1}{\sqrt{2\pi}}, \frac{1}{\sqrt{\pi}}\cos(nx), \frac{1}{\sqrt{\pi}}\sin\left(\left(n - \frac{1}{2} \right)x \right) \right\}_{n=1}^{\infty}.
	\end{align*}
\end{block}


\end{frame}


\begin{frame}{Orthogonality of Eigenfunctions}

\begin{block}{}
	\justifying
	\structb{Step 3.} \highlightg{Solve this ODE and obtain eigenvalues \& eigenfunctions.}
\end{block}
\begin{block}{}
	\justifying
	\structb{Example 5.} The Sturm-Liouville problem can have various forms deciding the eigenfunctions.
	\begin{itemize}
		\item Wave equation $c^2 u_{xx} = u_{tt}, 0 < x < l, t > 0$ gives two ODEs
		\begin{align*}
		X'' + \lambda X = 0,\qquad T'' + \lambda c^2 T = 0.
		\end{align*}
		\item The equation for the suspended chain $u_{tt} = g\cdot xu_x$ yields
		\begin{align*}
		(xX')' + \lambda X = 0\quad\Rightarrow \quad y^2Y'' + yY' + \lambda y^2Y = 0, \\
		T'' + \lambda gT = 0.
		\end{align*}
	\end{itemize}
\end{block}


\end{frame}





\subsection{Fit into Initial Conditions}

\begin{frame}{Fit into Initial Conditions}

\begin{block}{}
	\justifying
	\structb{Step 5.} \highlightg{Fit the general solution into the initial conditions.}
\end{block}

\begin{block}{}
	\justifying
	Expand the initial condition using the orthonormal system formed by the eigenfunctions.
\end{block}




\end{frame}



\begin{frame}{Fit into Initial Conditions}

The Fourier series for $L^2([0, L])$:
\begin{itemize}
	\item \structb{The Fourier-Euler Basis.}
	$$
	\mathcal{B}_1 := \left\{\frac{1}{\sqrt{L}}, \sqrt{\frac{2}{L}}\cos\left(\frac{2\pi nx}{L} \right), \sqrt{\frac{2}{L}}\sin\left(\frac{2\pi nx}{L} \right) \right\}^{\infty}_{n=1}
	$$
	\item \structb{The Fourier-Cosine Basis.}
	$$
	\mathcal{B}_2 := \left\{\frac{1}{\sqrt{L}}, \sqrt{\frac{2}{L}}\cos\left(\frac{\pi nx}{L} \right) \right\}^{\infty}_{n=1}
	$$
	\item \structb{The Fourier-Sine Basis.}
	$$
	\mathcal{B}_3 := \left\{\sqrt{\frac{2}{L}}\sin\left(\frac{\pi nx}{L} \right) \right\}^{\infty}_{n=1}
	$$
\end{itemize}


\end{frame}



\begin{frame}{Fit into Initial Conditions}

\begin{block}{}
	\begin{itemize}
		\item \structb{The complex Fourier-Euler Basis.}
		$$
		\mathcal{B}_{\mathcal{F}} = \left\{\frac{1}{\sqrt{2L}} e^{inx\pi/L} \right\}_{n=-\infty}^{\infty}
		$$
		\item\structb{The Bessel Functions.}
		\begin{align*}
		\left\{\frac{1}{\sqrt{L}|J'_n(\alpha_{n, m})|} J_n(\alpha_{n, m}\sqrt{x/L}) \right\}_{m=1}^{\infty}
		\end{align*}
	\end{itemize}
\end{block}


\end{frame}





\section{Final Remarks}


\begin{frame}{Final Remarks}


\begin{itemize}
	\justifying
	\item The choice of eigenvalues need to be discussed in order to satisfy the boundary conditions.
	\item Scale the basis functions according to the length of the interval.
	\item The expression of eigenfunctions can be either as sine/cosine functions or exponential functions. CHOOSE SMARTLY :)
\end{itemize}


\end{frame}




\begin{frame}{}
  
  \centering
  \highlightr{Good luck for your Final!}
  
  
\end{frame}

\end{document}


